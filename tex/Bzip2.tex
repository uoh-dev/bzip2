\documentclass{article}
%\usepackage[margin=1.3in]{geometry}
\usepackage[margin=1.84in, top=4cm]{geometry}

\date{Abgabe: \today}
\author{Florian Brohm, 7443251 \and Toprak Saricerci, 7445073}

\title{Seminar: Datenkompression\\Bzip2}

\usepackage{amssymb}
\usepackage{amsmath}
\usepackage{amsthm}
\usepackage{stmaryrd}
\usepackage{wasysym}
\usepackage{xcolor}
\usepackage{graphicx}
\usepackage[ngerman]{babel}
%\usepackage{changepage}
%\usepackage[thinc]{esdiff}
%\usepackage[makeroom]{cancel}
%\usepackage{pdfpages}

%\usepackage{tikz}
%\usepackage{listings}
%\usepackage{tabularx}
\usepackage{float}
\usepackage[font=small]{caption}

%\usetikzlibrary{arrows,automata}

\usepackage[sorting=none]{biblatex}
\addbibresource{quellen.bib}

\newcommand{\bla}{\bigwedge\limits}
\newcommand{\blo}{\bigvee\limits}

\newcommand{\la}{\land}
\newcommand{\lo}{\lor}
\newcommand{\lr}{\rightarrow}
\newcommand{\llr}{\leftrightarrow}
\newcommand{\Llr}{\Leftrightarrow}
\newcommand{\lp}{\oplus}
\newcommand{\p}{\text{potenz}}
\newcommand{\R}{\Rightarrow}
\newcommand{\n}[1]{\overline{{#1}}}
\newcommand{\bb}[1]{\mathbb{{#1}}}

\newcommand{\step}[1]{&& \left|\ {#1} \right.}
\newcommand{\e}[2]{{#1}\cdot 10^{{#2}}}
\newcommand{\blue}{\textcolor{blue}}
\newcommand{\red}{\textcolor{red}}
\newcommand{\quelle}{\red{quelle }}
\newcommand{\todo}{\red{todo }}
\newcommand{\dunderline}[1]{\underline{\underline{#1}}}
\newcommand{\reff}[2]{\hyperref[{#2}]{{#1}\ref*{#2}}}


\renewcommand{\labelenumi}{\alph{enumi})}
\renewcommand{\labelenumii}{\roman{enumii})}
\setlength\parindent{0pt}


\usepackage{hyperref}
\hypersetup{
	colorlinks,
	linkcolor={blue},
	urlcolor={red}, 
    citecolor={blue}
}

\begin{document}
\maketitle
\newpage

{
    \hypersetup{linkcolor=blue}
    \tableofcontents
}
\listoftables
\listoffigures
\newpage
\section{Was ist Bzip2?}
\section{Was ist eine Enkodierung?}
\newpage
\section{Run-length encoding}
\subsection{Enkodierung}
\subsubsection{Implementierung}
\subsection{Dekodierung}
\subsubsection{Implementierung}
\newpage
\section{Huffman encoding}
\subsection{Enkodierung}
\subsubsection{Implementierung}
\subsection{Dekodierung}
\subsubsection{Implementierung}
\newpage
\section{Was ist eine Transformation?}
\section{Move-to-front transform}
\subsection{Transformation}
\subsubsection{Implementierung}
\subsection{Inverse Transformation}
\subsubsection{Implementierung}
\newpage
\section{Burrows-Wheeler transform}
\subsection{Transformation}
\subsubsection{Implementierung}
\subsection{Inverse Transformation}
\subsubsection{Implementierung}
\newpage
\section{Ergebnisse}
\subsection{Einfluss der Schritte auf die Kompression}
\printbibliography[heading=bibintoc]
\end{document}